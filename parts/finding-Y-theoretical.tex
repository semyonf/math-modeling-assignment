\part{НАХОЖДЕНИЕ $y^T$}
    \section{Составление системы дифференциальных уравнений}
        $$
            W(s) = \frac{k(1-as)}{(1+as)(1+b_1s+b_2s^2)} = \frac{P(s)}{Q(s)}
        $$
        \vspace{1cm}
        $$
            \begin{cases}
                \dfrac{1}{Q(s)} = \dfrac{u(s)}{x(s)}
                \\\\
                P(s) = \dfrac{y(s)}{u(s)}
            \end{cases}
            \Rightarrow
            \begin{cases}
                y(s) = k(1-as) \cdot u(s)
                \\
                x(s) = (1+as)(1+b_1s+b_2s^2) \cdot u(s)
            \end{cases}
        $$
        \vspace{1cm}
        $$
            \begin{aligned}
                y(s)
                &= k(1-as) \cdot u(s) \\
                &= (k-kas) \cdot u(s) \\
                &= ku(s) - kasu(s)
                \\\\
                x(s)
                &= (1+b_1s + b_2s^2 + as + b_1sas + asb_2s^2) \cdot u(s) \\
                &= u(s) + b_1su(s) + b_2s^2u(s) + asu(s) + b_1as^2u(s) + as^3b_2u(s)
            \end{aligned}
        $$
        \vspace{1cm}
        $$
            \begin{matrix}
                \begin{matrix}
                    x(s) \to x(t) \\
                    y(s) \to y(t) \\
                    u(s) \to u(t) \\
                \end{matrix}
                &
                S = \dfrac{d}{dt}
                &
                S^n = \dfrac{d^n}{dt^n}
            \end{matrix}
        $$
        \vspace{1cm}
        $$
            \begin{aligned}
                y(t) &= ku(t) - ka\dot{u}(t)
                \\
                x(t) &= u(t) + b_1\dot{u}(t) + b_2\ddot{u}(t) + a\dot{u}(t) + ab_1\ddot{u}(t) + ab_2\dddot{u}(t)
            \end{aligned}
        $$
        \vspace{1cm}
        $$
            \begin{aligned}
                \dddot{u}(t)
                &= \frac{1}{ab_2} \cdot (x(t) - u(t) - b_1\dot{u}(t) - b_2\ddot{u}(t) - a\dot{u}(t) - ab_1\ddot{u}(t))
                \\
                &= \frac{1}{ab_2} \cdot \underbrace{(x(t) - u(t) - (b_1 + a)\dot{u}(t) - (b_2 + ab_1)\ddot{u}(t))}_{\Omega}
            \end{aligned}
        $$

    \section{Замена переменных в системе дифференциальных уравнений}
        $$
            \begin{matrix}
                \begin{aligned}
                    z_1(t) = &u(t) &
                    \\
                    z_2(t) = &\dot{u}(t) = \dot{z_1}(t) &
                    \\
                    z_3(t) = &\ddot{u}(t) = \dot{z_2}(t) &
                    \\
                    &\dddot{u}(t) = \dot{z_3}(t) = \Omega &
                \end{aligned}
                &
                \Rightarrow
                &
                \begin{aligned}
                    \dot{z_1} &= z_2
                    \\
                    \dot{z_2} &= z_3
                    \\
                    \dot{z_3} &= \dfrac{\Omega}{ab_2}
                \end{aligned}
                &
                \Rightarrow
                &
                y(t) = kz_1 - kaz_2
            \end{matrix}
        $$

    \section{Решение системы методом Эйлера}
        $$
            \begin{cases}
                z_{1,n+1} = z_{1,n} + h \cdot z_{2,n}
                \\
                z_{2,n+1} = z_{2,n} + h \cdot z_{3,n}
                \\
                z_{3,n+1} = z_{3,n} + h \cdot \Omega
            \end{cases}
        $$
        $$
            y(s) = kz_{1,n} - kaz_{2,n} = k(z_{1,n} - az_{2,n})
        $$
    \newpage
    \section{График и таблица значений}
        \begin{center}
            \begin{tikzpicture}
                \begin{axis}[
                    width=16cm,
                    height=10cm,
                    xlabel=t,
                    ylabel=y(t),
                    minor tick num = 1,
                    grid = both
                ]
                    \addplot table [col sep=comma] {data/theor.csv};
                \end{axis}
            \end{tikzpicture}
            \pgfplotstabletypeset[
                col sep=comma,
                columns/x/.style = {
                    string type,
                    column name = $t$,
                    fixed zerofill,
                    precision=1,
                    dec sep align
                    },
                columns/y/.style = {
                    column name=$y(t)$,
                    fixed,
                    fixed zerofill,
                    precision=11,
                    dec sep align
                }
            ]{data/theor.csv}
        \end{center}

    \section{Проверка}
        При $x(t) = const (C) = 5$ и $k = 4$
        $$
            \lim_{s\to0}CW(s) = C \cdot \lim_{s\to0}\left(\dfrac{k\overbrace{(1-as)}^{\to1}}{\underbrace{(1+as)}_{\to1}\underbrace{(1+b_1s+b_2s^2)}_{\to1}}\right) = C \cdot k = 4 \cdot 5 = 20
        $$

    \section{Выводы первой части}
        Полученная функция сходится к значению того предела, который был вычислен. Был выбран период наблюдения в 25 точек, так как функция сходится и выполаживается быстро и нет смысла увеличивать их количество и/или размер шага.
