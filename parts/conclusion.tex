\newpage
\part*{Вывод}
    В ходе выполнения работы мною были освоены навыки разложения сложной передаточной функции в прозведение передаточных функций, включенных последовательно, перевод в вещественную форму, составление системы дифференциальных уравнений и решение её методом Эйлера. Далее удалось успешно построить график функции, выполаживающейся после 25 точек на верном значении предела.

    Также мной был реализована программа на языке Fortran, содержащая функционал для оптимизации функций методом Гаусса-Зейделя по разработанному ранее алгоритму. Работоспособность программы оптимизации была проверена на функциях эллипса и Розенброка.

    Следующим шагом был процесс формирование шума, который включает в себя разработку генератора случайных чисел согласно заданию, были построены гистограммы распределения значений в разных диапазонах.

    Предыдущие шаги позволили провести оптимизацию трех экспериментальных функций с дальнейшим нахождением неизвестных.

    Знания, полученные в рамках предмета "Математические модели" и данной курсовой работы в частности, позволили мне приобрести понимание подходов этой области математики и будут полезны в дальнейшей работе.
