\thispagestyle{empty}

\begin{center}
    Санкт-Петербургский политехнический университет Петра Великого\\
    Институт Компьютерных Наук и Технологий\\
    \bfseries{Высшая школа программной инженерии}
\end{center}

\vspace{20ex}

\begin{center}
    {
    \LARGE \textbf{КУРСОВАЯ РАБОТА} \\[3ex]
    по дисциплине: «Математические модели» \\
    по теме: «Моделирование процесса параметрической идентификации динамического объекта» \\[3ex]
    Вариант №39
    }
\end{center}

\vspace{40ex}

\noindent Выполнил\\
студент гр.23531/21\hfill
\begin{minipage}{0.7\textwidth}
    \hfill \uline{\hspace{3cm}} \hspace{1.1cm} С.А. Фомин
\end{minipage}

\vspace{3ex}

\noindent Руководитель,\\
доцент\hfill
\begin{minipage}{0.7\textwidth}
    \hfill \uline{\hspace{3cm}} \hspace{0.5cm} Т.В. Леонтьева
\end{minipage}

\vspace{3ex}

\hfill \begin{minipage}{0.6\textwidth} \hfill «\uline{\hspace{1cm}}»\uline{\hspace{3cm}} 2017 г.\end{minipage}

\vfill

\begin{center}
    Санкт-Петербург\\
    2017
\end{center}

\newpage
\thispagestyle{empty}
\begin{center}
    ЗАДАНИЕ\\
    НА ВЫПОЛНЕНИЕ КУРСОВОЙ РАБОТЫ \\[1ex]
    студенту группы 23531/21 Фомину С.А.
\end{center}

\vspace{0.5cm}

\begin{enumerate}
    \item
        \textbf{Тема работы:} Моделирование процесса параметрической идентификации динамического объекта, вариант №39
    \item
        \textbf{Срок сдачи студентом законченного проекта (работы):} 30.12.2017
    \item
        \textbf{Исходные данные к проекту (работе):}
            $$W(s) = \frac{k(1-as)}{(1+as)(1+b_1s+b_2s^2)}$$
            $$
                \begin{matrix}
                    \begin{aligned}
                        x(t) = 5 \\
                        k = 4 \\
                        a = 3 \\
                        b_1 = 1 \\
                        b_2 = 2
                    \end{aligned}
                    &
                    \hspace{3cm}
                    \begin{aligned}
                        \text{метод оптимизации}&: \text{Гаусса-Зейделя $(GZ_1)$} \\
                        \text{целевая функция}&: \text{2} \\
                        \text{шум}&: \text{2} \\
                        \text{считать неизвестными}&: \text{$k, b_1$} \\
                    \end{aligned}
                \end{matrix}
            $$
    \item
        \textbf{Содержание пояснительной записки (перечень подлежащих разработке вопросов):} введение, основная часть (раскрывается структура основной части), заключение, список использованных источников, приложения, нахождение $y^T$, создание и проверка ГСЧ с помощью гистограмм, формирование шума и $y^\epsilon$, реализация алгоритма оптимизации и проверка на тостовых функциях, нахождение $m^M$ и целевой функции $(CF)$, её оптимизация. Примерный объем пояснительной записки 28 страниц машинописного текста
    \item
        \textbf{Консультанты:} Леонтьева Т.В.
    \item
        \textbf{Дата получения задания:} 05 сентября 2017 г.
\end{enumerate}

\vfill

\noindent \textbf{Руководитель} \hfill
\hfill \uline{\hspace{3cm}} \hspace{1.1cm} Леонтьева Т.В.

\vspace{1cm}

\noindent \textbf{Задание принял к исполнению} \hfill
\hfill \uline{\hspace{3cm}} \hspace{1.7cm} Фомин С.А.

\newpage
