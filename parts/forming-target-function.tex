\part{ФОРМИРОВАНИЕ ЦЕЛЕВОЙ ФУНКЦИИ}
    \begin{center}
        Целевая функция 2: $CF=\sum(Y_{\text{э}} - Y_{\text{м}})^2$
    \end{center}

    Неизвестными являются параметры $k$, $b_1$, для оптимизации целевой функции к ней применяется ранее написанный алгоритм. На каждой итерации параметры для расчета нового $y_M$ будут изменяться. Значения остальных переменных остаются такими же, как при вычислении теоретических значений.

    Оптимизация будет выполняться для трех экспериментальных функций $y_{\text{Э1}}, y_{\text{Э2}}, y_{\text{Э3}}$. В результате получаем три пары значений неизвестных параметров.

    \begin{center}
        \begin{tabular}{c|c|c|}
        \cline{2-3}
                                    & k    & $b_1$  \\ \hline
        \multicolumn{1}{|c|}{$CF1$} & 3.9848 & 0.9650 \\ \hline
        \multicolumn{1}{|c|}{$CF2$} & 4.0125 & 1.0125 \\ \hline
        \multicolumn{1}{|c|}{$CF3$} & 3.9492 & 0.8227  \\ \hline
        \multicolumn{1}{|c|}{$Y_{\text{сред}}$} & 3.9822  & 0.9333   \\ \hline
        \end{tabular}
    \end{center}

    \begin{center}
        \begin{tikzpicture}
            \begin{axis}[
                width=16cm,
                height=10cm,
                samples=1,
                minor tick num = 1,
                grid = both,
                legend style={at={(0.5,-0.1)},anchor=north}
            ]

            \addplot table [col sep=comma] {data/final.csv};
            \addlegendentry{$Y_{\text{сред.}}$}
            \addplot table [col sep=comma] {data/theor.csv};
            \addlegendentry{$Y_{\text{теор.}}$}

            \end{axis}
        \end{tikzpicture}
    \end{center}

    Так как график, построенный по теоретическим данным совпадает с полученным графиком функции с модельными значениями, можно сделать вывод, что оптимизация выполнена верно.
