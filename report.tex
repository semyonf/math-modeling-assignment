\documentclass[a4paper, 12pt]{article}

\usepackage[T2A]{fontenc}
\usepackage[utf8]{inputenc}
\usepackage[english,russian]{babel}
\usepackage{listings}
\usepackage{multirow}
\usepackage{pgfplots}
\usepackage{pgfplotstable}
\usepackage[normalem]{ulem}
\usepackage{geometry}
\usepackage{amsmath}
\usepackage{titlesec}

\usepgfplotslibrary{units}
\pgfplotsset{compat=1.14}
\geometry{left=25mm, right=10mm, top=20mm, bottom=20mm}

\renewcommand{\thepart}{\arabic{part}}
\renewcommand{\thesection}{\thepart.\arabic{section}}

\titleformat{\part}
    {\normalsize\bfseries}
    {ЧАСТЬ \thepart.}
    {1mm}{}

\titleformat{\section}
    {\normalsize\bfseries}
    {\thesection.}
    {1mm}{}

\title{3.1 Титульный лист}

\begin{document}

\begin{tikzpicture}
    \begin{axis}
        \addplot table [col sep=comma] {data/theor.csv};
    \end{axis}
\end{tikzpicture}

\begin{tikzpicture}
    \begin{axis}
        \addplot table [col sep=comma] {data/noised005.csv};
    \end{axis}
\end{tikzpicture}

\begin{tikzpicture}
    \begin{axis}
        \addplot table [col sep=comma] {data/theor.csv};
        \addplot table [col sep=comma] {data/optimized.csv};
    \end{axis}
\end{tikzpicture}

\thispagestyle{empty}

\begin{center}
    Санкт-Петербургский политехнический университет Петра Великого\\
    Институт Компьютерных Наук и Технологий\\
    \bfseries{Высшая школа программной инженерии}
\end{center}

\vspace{20ex}

\begin{center}
    {
    \LARGE \textbf{КУРСОВАЯ РАБОТА} \\[3ex]
    по дисциплине: «Математические модели» \\
    по теме: «Моделирование процесса параметрической идентификации динамического объекта» \\[3ex]
    Вариант №39
    }
\end{center}

\vspace{40ex}

\noindent Выполнил\\
студент гр.23531/21\hfill
\begin{minipage}{0.7\textwidth}
    \hfill \uline{\hspace{3cm}} \hspace{1.1cm} С.А. Фомин
\end{minipage}

\vspace{3ex}

\noindent Руководитель,\\
доцент\hfill
\begin{minipage}{0.7\textwidth}
    \hfill \uline{\hspace{3cm}} \hspace{0.5cm} Т.В. Леонтьева
\end{minipage}

\vspace{3ex}

\hfill \begin{minipage}{0.6\textwidth} \hfill «\uline{\hspace{1cm}}»\uline{\hspace{3cm}} 2017 г.\end{minipage}

\vfill

\begin{center}
    Санкт-Петербург\\
    2017
\end{center}

\newpage
\thispagestyle{empty}
\begin{center}
    ЗАДАНИЕ\\
    НА ВЫПОЛНЕНИЕ КУРСОВОЙ РАБОТЫ \\[1ex]
    студенту группы 23531/21 Фомину С.А.
\end{center}

\vspace{0.5cm}

\begin{enumerate}
    \item
        \textbf{Тема работы:} Моделирование процесса параметрической идентификации динамического объекта, вариант №39
    \item
        \textbf{Срок сдачи студентом законченного проекта (работы):} 30.12.2017
    \item
        \textbf{Исходные данные к проекту (работе):}
            $$W(s) = \frac{k(1-as)}{(1+as)(1+b_1s+b_2s^2)}$$
            $$
                \begin{matrix}
                    \begin{aligned}
                        x(t) = 5 \\
                        k = 4 \\
                        a = 3 \\
                        b_1 = 1 \\
                        b_2 = 2
                    \end{aligned}
                    &
                    \hspace{3cm}
                    \begin{aligned}
                        \text{метод оптимизации}&: \text{Гаусса-Зейделя $(GZ_1)$} \\
                        \text{целевая функция}&: \text{2} \\
                        \text{шум}&: \text{2} \\
                        \text{считать неизвестными}&: \text{$k, b_1$} \\
                    \end{aligned}
                \end{matrix}
            $$
    \item
        \textbf{Содержание пояснительной записки (перечень подлежащих разработке вопросов):} введение, основная часть (раскрывается структура основной части), заключение, список использованных источников, приложения, нахождение $y^T$, создание и проверка ГСЧ с помощью гистограмм, формирование шума и $y^\epsilon$, реализация алгоритма оптимизации и проверка на тостовых функциях, нахождение $m^M$ и целевой функции $(CF)$, её оптимизация. Примерный объем пояснительной записки 28 страниц машинописного текста
    \item
        \textbf{Консультанты:} Леонтьева Т.В.
    \item
        \textbf{Дата получения задания:} 05 сентября 2017 г.
\end{enumerate}

\vfill

\noindent \textbf{Руководитель} \hfill
\hfill \uline{\hspace{3cm}} \hspace{1.1cm} Леонтьева Т.В.

\vspace{1cm}

\noindent \textbf{Задание принял к исполнению} \hfill
\hfill \uline{\hspace{3cm}} \hspace{1.7cm} Фомин С.А.

% \newpage
% \tableofcontents

\newpage
\part{НАХОЖДЕНИЕ $y^T$ (ТЕОРЕТИЧЕСКОГО)}
\section{Составление системы дифференциальных уравнений}

$$
    W(s) = \frac{k(1-as)}{(1+as)(1+b_1s+b_2s^2)} = \frac{P(s)}{Q(s)}
$$

\vspace{1cm}

$$
    \begin{cases}
        \dfrac{1}{Q(s)} = \dfrac{u(s)}{x(s)}
        \\\\ % Это зашквар, надо будет поправить
        P(s) = \dfrac{y(s)}{u(s)}
    \end{cases}
    \Rightarrow
    \begin{cases}
        y(s) = k(1-as) \cdot u(s)
        \\
        x(s) = (1+as)(1+b_1s+b_2s^2) \cdot u(s)
    \end{cases}
$$

\vspace{1cm}

$$
    \begin{aligned}
        y(s)
        &= k(1-as) \cdot u(s) \\
        &= (k-kas) \cdot u(s) \\
        &= ku(s) - kasu(s)
        \\\\
        x(s)
        &= (1+b_1s + b_2s^2 + as + b_1sas + asb_2s^2) \cdot u(s) \\
        &= u(s) + b_1su(s) + b_2s^2u(s) + asu(s) + b_1as^2u(s) + as^3b_2u(s)
    \end{aligned}
$$

\vspace{1cm}

$$
    \begin{matrix}
        \begin{matrix}
            x(s) \to x(t) \\
            y(s) \to y(t) \\
            u(s) \to u(t) \\
        \end{matrix}
        &
        S = \dfrac{d}{dt}
        &
        S^n = \dfrac{d^n}{dt^n}
    \end{matrix}
$$

\vspace{1cm}

$$
    \begin{aligned}
        y(t) &= ku(t) - ka\dot{u}(t)
        \\
        x(t) &= u(t) + b_1\dot{u}(t) + b_2\ddot{u}(t) + a\dot{u}(t) + ab_1\ddot{u}(t) + ab_2\dddot{u}(t)
    \end{aligned}
$$

\vspace{1cm}

$$
    \begin{aligned}
        \dddot{u}(t)
        &= \frac{1}{ab_2} \cdot (x(t) - u(t) - b_1\dot{u}(t) - b_2\ddot{u}(t) - a\dot{u}(t) - ab_1\ddot{u}(t))
        \\
        &= \frac{1}{ab_2} \cdot \underbrace{(x(t) - u(t) - (b_1 + a)\dot{u}(t) - (b_2 + ab_1)\ddot{u}(t))}_{\Omega}
    \end{aligned}
$$

\section{Замена переменных в системе дифференциальных уравнений}
$$
    \begin{matrix}
        \begin{aligned}
            z_1(t) = &u(t) &
            \\
            z_2(t) = &\dot{u}(t) = \dot{z_1}(t) &
            \\
            z_3(t) = &\ddot{u}(t) = \dot{z_2}(t) &
            \\
            &\dddot{u}(t) = \dot{z_3}(t) = \Omega &
        \end{aligned}
        &
        \Rightarrow
        &
        \begin{aligned}
            \dot{z_1} &= z_2
            \\
            \dot{z_2} &= z_3
            \\
            \dot{z_3} &= \dfrac{\Omega}{ab_2}
        \end{aligned}
        &
        \Rightarrow
        &
        y(t) = kz_1 - kaz_2
    \end{matrix}
$$

\section{Решение системы методом Эйлера}
$$
    \begin{cases}
        z_{1,n+1} = z_{1,n} + h \cdot z_{2,n}
        \\
        z_{2,n+1} = z_{2,n} + h \cdot z_{3,n}
        \\
        z_{3,n+1} = z_{3,n} + h \cdot \Omega
    \end{cases}
$$

$$
    y(s) = kz_{1,n} - kaz_{2,n} = k(z_{1,n} - az_{2,n})
$$

\section{График и таблица значений}

\pgfplotsset{
    gr1/.style = {
        red,
        samples = 100,
        mark = *,
        mark options = {
            scale = 0.4
        },
        line width = 0.5mm
    }
}

\begin{center}


    % НАДО ЕЩЕ ТАБЛИЦУ ЗНАЧЕНИЙ

    % \begin{tikzpicture}
    %     \begin{axis}[
    %         width=16cm,
    %         height=10cm,
    %         xlabel=t,
    %         ylabel=y(t),
    %         samples=1,
    %         minor tick num = 1,
    %         grid = both
    %     ]

    %     \end{axis}
    % \end{tikzpicture}

    \begin{tikzpicture}
        \begin{axis}
            \addplot table [col sep=comma] {data/theor.csv};
        \end{axis}
    \end{tikzpicture}

    \vspace{1cm}

    \pgfplotstabletypeset[
        col sep=comma,
        columns/x/.style = {
            string type,
            column name = $t$,
            fixed zerofill,
            precision=1,
            dec sep align
            },
        columns/y/.style = {
            column name=$y(t)$,
            fixed,
            fixed zerofill,
            precision=11,
            dec sep align
        }
    ]{data/theor.csv}
\end{center}

\newpage

\section{Проверка}

При $x(t) = const (C) = 5$ и $k = 4$

$$
    \lim_{s\to0}CW(s) = C \cdot \lim_{s\to0}\left(\dfrac{k\overbrace{(1-as)}^{\to1}}{\underbrace{(1+as)}_{\to1}\underbrace{(1+b_1s+b_2s^2)}_{\to1}}\right) = C \cdot k = 4 \cdot 5 = 20
$$

\section{Выводы первой части}
По полученным данным можно сделать вывод, что полученные значения совпадают с ожидаемыми, полученная функция сходится к значению того пределе, который был вычислен, был выбран период наблюдения в 25 точек, что функция сходится и приходит к периоду покоя быстро и нет смысла увеличивать их количество.

\part{ФОРМИРОВАНИЕ ШУМА}
\setcounter{section}{0}
\section{Описание процесса формирования шума}
Интервал распределения шума задан как $ (-\Delta y; \Delta y) $, где коэффициент $\Delta y$ вычисляется по формуле:
$$
    \Delta y = max(y^{T}) \cdot
    \begin{Bmatrix}
        0,05 \\
        0,1 \\
        0,2
    \end{Bmatrix}
    =
    \begin{Bmatrix}
        1.092 \\
        2.184 \\
        4.367
    \end{Bmatrix}
$$

После вычисления значений шума, необходимо добавить их к теоретическим значениям в интересующих нас точках для получения экспериментальных значений.

Можно составить блок-схему процесса генерации экспериментальных значений.
\end{document}
